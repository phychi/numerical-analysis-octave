\documentclass[12pt, a4paper]{article} % 使用 article 文档类,12磅字体,A4纸

% ===== 预定义包 =====
\usepackage[UTF8]{ctex} % 支持中文
\usepackage{geometry} % 设置页边距
\geometry{left=2.5cm, right=2.5cm, top=2.5cm, bottom=2.5cm}
\usepackage{amsmath, amssymb, amsthm} % 数学符号和定理环境
\usepackage{booktabs} % 制作三线表
\usepackage{enumitem} % 控制列表格式
\usepackage{hyperref} % 创建超链接
\usepackage{titlesec} % 自定义标题格式
\usepackage{multirow} % 表格中合并行

% ===== 自定义格式 =====
% 设置章节标题格式
\titleformat{\section}[block]{\Large\bfseries\centering}{\thesection}{1em}{}
\titleformat{\subsection}[block]{\large\bfseries}{\thesubsection}{1em}{}

% 设置列表间隔
\setlist[itemize]{noitemsep, topsep=0pt} % 无间隔的项目列表
\setlist[enumerate]{noitemsep, topsep=0pt} % 无间隔的编号列表

% ===== 文档开始 =====
\begin{document}
	
	
	\section{学时分配总览}
	\begin{table}[h!]
		\centering
		\caption{课程内容与学时分配表}
		\label{tab:schedule}
		\begin{tabular}{c l c}
			\toprule
			\textbf{部分} & \textbf{内容} & \textbf{学时} \\
			\midrule
			I & 绪论与基础 & 4 \\
			II & 非线性方程的数值解法 & 6 \\
			III & 线性方程组的数值解法 & 8 \\
			IV & 插值方法与数值逼近 & 8 \\
			V & 数值积分 & 5 \\
			VI & 矩阵特征值与特征向量的计算 & 5 \\
			VII & 常微分方程初值问题的数值解法 & 8 \\
			VIII & 课程复习与综合应用 & 4 \\
			\midrule
			& \textbf{总计} & \textbf{48} \\
			\bottomrule
		\end{tabular}
	\end{table}
	
	% ===== 详细教学内容 =====
	\section{详细教学内容}
	
	\subsection{第一部分:绪论与基础 (4学时)}
	\begin{enumerate}
		\item \textbf{第1-2学时:课程导论与误差分析}
		\begin{itemize}
			\item 数值分析的研究对象与意义(科学与工程中的计算问题举例)。
			\item 误差的来源:模型误差、观测误差、截断误差、舍入误差。
			\item 误差的基本概念:绝对误差、相对误差、有效数字。
			\item 数值计算的稳定性与病态问题。\textbf{条件数}的概念(以线性方程组为例)。
			\item \textit{重点难点:} 理解条件数是问题本身固有的属性,与算法无关。
		\end{itemize}
		
		\item \textbf{第3-4学时:浮点数系统与数值计算中的原则}
		\begin{itemize}
			\item IEEE浮点数标准简介(舍入、上溢、下溢)。
			\item 数值计算中应注意的若干问题(避免相近数相减、防止大数吃小数、注意除法分母、简化计算步骤)。
			\item (可选)算法复杂度的简单介绍($O(n)$ notation)。
			\item \textbf{实践环节:} 演示一个由于数值不稳定导致计算失败的经典案例(如:递归计算积分)。
		\end{itemize}
	\end{enumerate}
	
	% --- 后续部分结构类似,为节省篇幅,此处提供模板,其余部分可依此格式编写 ---
	\subsection{第二部分:非线性方程的数值解法 (6学时)}
	\begin{enumerate}
		\item \textbf{第5-6学时:非线性方程求根}
		\begin{itemize}
			\item \textbf{二分法:}思想、算法、收敛性分析(线性收敛)、误差估计。
			\item \textbf{不动点迭代法:}构造迭代格式,收敛性定理(局部收敛性、压缩映射原理)。
			\item \textit{重点难点:} 理解不动点迭代的收敛条件与收敛阶的定义。
		\end{itemize}
		
		\item \textbf{第7-8学时:高效求根算法}
		\begin{itemize}
			\item \textbf{Newton法:}几何推导、迭代公式、局部平方收敛性。
			\item \textbf{割线法:}思想、迭代公式、超线性收敛性(收敛阶为黄金比例)。
			\item \textit{重点难点:} Newton法对初值的敏感性以及导数难以获取时的处理。
		\end{itemize}
		
		\item \textbf{第9-10学时:非线性方程组的Newton法简介与算法比较}
		\begin{itemize}
			\item 将Newton法推广到非线性方程组情形,介绍其矩阵形式。
			\item 各种方法的比较:收敛速度、计算成本、robustness。
			\item (可选)MATLAB/Python中 \texttt{fzero}, \texttt{fsolve} 函数的使用。
			\item \textbf{实践环节:} 编程实现Newton法求解一个非线性方程,并绘制收敛过程。
		\end{itemize}
	\end{enumerate}
	
	% ... (第三至第七部分请按照相同的格式在此处继续填写) ...
	
	\subsection{第八部分:课程复习与综合应用 (4学时)}
	\begin{enumerate}
		\item \textbf{第45-48学时:课程复习与综合应用}
		\begin{itemize}
			\item \textbf{课程总复习(2学时):} 梳理各章节知识脉络,强调不同领域算法之间的联系与对比。
			\item \textbf{课程项目展示与答疑(2学时):} 学生分组展示其课程项目的成果,交流学习心得,教师进行点评和总结。
			\item \textbf{前沿简介(可选):} 简要介绍当今数值分析领域的热点,如有限元法、快速多极子算法、随机数值方法等,激发学生进一步学习的兴趣。
		\end{itemize}
	\end{enumerate}
	
	% ===== 教学建议 =====
	\section*{教学建议}
	\begin{itemize}
		\item \textbf{理论联系实际:} 每个章节都应配有从物理、工程、金融等领域引出的实际案例。
		\item \textbf{编程实践:} 鼓励学生使用MATLAB或Python(SciPy库)实现核心算法,这不仅加深理解,也培养了必备的科研技能。
		\item \textbf{批判性思维:} 引导学生不仅仅“会算”,更要“会选”和“会评”,即能根据问题特性选择最合适的算法,并能批判性地分析计算结果的可靠性。
		\item \textbf{利用现代工具:} 教学中可以演示如何使用专业的数值计算软件/库来高效解决问题,让学生了解“造轮子”和“用轮子”的平衡。
	\end{itemize}
	
\end{document}